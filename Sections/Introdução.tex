\section{Introdução}

Com a evolução das tecnologias de comunicação sem fios, garantir uma boa qualidade de sinal está a tornar-se um problema cada vez mais complexo. Especialmente no contexto de otimização de espetro e de energia. 

Este trabalho foca-se na análise de cobertura de sinal e qualidade em sistemas de transmissão sem fios que usam modulações de multi e mono-portadora, OFDM e SC-FDE. Estas duas técnicas são muito utilizadas em normas de comunicação sem fio, tendo ambas diferentes vantagens e desvantagens. 

Os objetivos principais a investigar são o efeito de relação sinal-ruído (SNR), atenuação do canal e do nível de ruído na cobertura e qualidade do sinal, comparação de performance entre OFDM e SC-FDE.

Este trabalho encontra-se dividido em duas fases.

Fase 1: Esta fase examina as densidades energéticas atingíveis, potência recebida e BER dependendo de potência, distância e condições do canal utilizando OFDM.

Fase 2: Esta fase explora desempenho do OFDM e SC-FDE com diferentes modulações e condições de amplificação.

\subsection{Problema da SNR}

A relação sinal-ruído é uma métrica em sistemas de comunicações sem fios que representa a relação da potência transmitida do sinal desejado e do ruído existente no canal.

Num sistema de comunicação existem várias condições que afetam o SNR, os fatores importantes a destacar são:

\begin{itemize}
    \item Atenuação do canal.
    \item Frequência da portadora.
    \item Seletividade do canal.
    \item Noise Floor.
\end{itemize}

Esses fatores determinam a potência mínima necessária no recetor para que a comunicação ainda seja viável. A atenuação e a frequência influenciam diretamente a potência recebida, sendo possível calcular o 
Loss Space Path Loss(LFSPL), através da equação \ref{eq:LFSPL}:

\begin{equation}
    \text{LFSPL (dB)} \approx  32.45 + 20 \cdot \log(f, \text{MHz}) + 20 \cdot \log(r, \text{km})
    \label{eq:LFSPL}
\end{equation}

E a potência recebida, equação \ref{eq:pr}:

\begin{equation}
    P_r = \text{EIRP} + G_r(\text{dB}) + 27.5 - 20\log(f, \text{MHz}) - 20\log(r, \text{meters})
    \label{eq:pr}
\end{equation}