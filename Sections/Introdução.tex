\section{Introdução}

Com a evolução das tecnologias de comunicação sem fios, garantir uma boa qualidade de sinal está a tornar-se um problema cada vez mais complexo. Especialmente no contexto de otimização de espetro e de energia. 

Este trabalho foca-se na análise de cobertura de sinal e qualidade em sistemas de transmissão sem fios que usam modulações de multi e mono-portadora, OFDM e SC-FDE. Estas duas técnicas são muito utilizadas em normas de comunicação sem fio, tendo ambas diferentes vantagens e desvantagens. 

Os objetivos principais a investigar são o efeito de relação sinal-ruído (SNR), atenuação do canal e do nível de ruído na cobertura e qualidade do sinal, comparação de performance entre OFDM e SC-FDE.

Este trabalho encontra-se dividido em duas fases.

Fase 1: Esta fase examina as densidades energéticas atingíveis, potência recebida e BER dependendo de potência, distância e condições do canal utilizando OFDM.

Fase 2: Esta fase explora desempenho do OFDM e SC-FDE com diferentes modulações e condições de amplificação usando Solid State Power Amplifiear (SSPA).

\subsection{Problema da SNR}

A relação sinal-ruído é uma métrica em sistemas de comunicações sem fios que representa a relação da potência transmitida do sinal desejado e do ruído existente no canal.

Num sistema de comunicação existem várias condições que afetam o SNR, os fatores importantes a destacar são:

\begin{itemize}
    \item Atenuação do canal.
    \item Frequência da portadora.
    \item Seletividade do canal.
    \item Noise Floor.
\end{itemize}

Esses fatores determinam a potência mínima necessária no recetor para que a comunicação ainda seja viável. A atenuação e a frequência influenciam diretamente a potência recebida, sendo possível calcular o 
Loss Space Path Loss(LFSPL), através da equação \ref{eq:LFSPL}:
\textcolor{red}{Mudar para expressões novas}

\begin{equation}
    \text{LFSPL (dB)} \approx  32.45 + 20 \cdot \log(f, \text{MHz}) + 20 \cdot \log(r, \text{km})
    \label{eq:LFSPL}
\end{equation}

E a potência recebida, equação \ref{eq:pr}:

\begin{equation}
    P_r = \text{EIRP} + G_r(\text{dB}) + 27.5 - 20\log(f, \text{MHz}) - 20\log(r, \text{meters})
    \label{eq:pr}
\end{equation}

\subsection{transmissão Multi-Portadora - OFDM}

O OFDM (Orthogonal Frequency Division Multiplexing) combate a seletividade do canal dividindo a banda em subportadoras ortogonais 
e utilizando um prefixo cíclico para eliminar a Interferência Inter-Símbolo (ISI), 
permitindo uma equalização simples no domínio da frequência.
 No entanto, sofre de um elevado PAPR (Peak-to-Average Power Ratio), exigindo 
 amplificadores lineares pouco eficientes, e é sensível a erros de 
 sincronização.

\subsection{Transmissão Mono-Portadora - SC-FDE}
Como alternativa para o uplink, o SC-FDE (Single Carrier Frequency Domain Equalization) mantém a estrutura de blocos 
e o prefixo cíclico do OFDM, mas desloca a IFFT para o recetor. 
Isto resulta em menores flutuações de envolvente (menor PAPR), 
reduzindo as exigências de linearidade na amplificação, embora constelações de ordem elevada (como 64-QAM e 256-QAM) ainda apresentem desafios de distorção.

\subsection{Amplificação}

A eficiência energética depende criticamente da amplificação. Ao contrário do 2G (envolvente constante), as modulações atuais (4G/5G) possuem envolventes variáveis, 
impedindo o uso de amplificadores saturados sem introduzir distorção severa.
Neste trabalho, modela-se o Solid State Power Amplifiear  (SSPA) através do Modelo de Rapp (Equação \ref{eq:Rapp}), analisando o impacto dos parâmetros de não-linearidade ($p=1$ e $p=2$) e parâmetros mais proximos da linearidade ($p=10$ e $p=50$)
no consumo energético e na necessidade de backoff  para manter a integridade do sinal.

\begin{equation}
    A(r) = \frac{r}{\left( 1 + \left( \frac{|r|}{S_{sat}} \right)^{2p} \right)^{\frac{1}{2P}}}
    \label{eq:Rapp}
\end{equation}

\subsection{Potência Radiada}

A evolução da potência radiada reflete as restrições de modulação explicadas acima. 
No 2G, a envolvente constante (GMSK) permitia o uso de amplificadores saturados e potências de 400 W com alcance de 30 km. 
Com o desenolvimento do 3G e 4G, as potências máximas caíram drasticamente passando para
250mW a 40W, devido isso houve necessidade de  usar amplificadores lineares (SSPA) para suportar modulações de amplitude variável sem distorção.

Nos dias de hoje o 5G inverte esta tendência de redução. 
Para garantir a elevada relação sinal-ruído (SNR) exigida por constelações densas e 
técnicas multi-portadora, a potência de emissão necessita de aumentar novamente, podendo atingir os 40 W por cadeia de transmissão.

