\section{Resultados}

\subsection{Fase 1}

\subsubsection{Resultados obtidos através de modelos matemáticos}

Como foi explicado no capítulo anterior, foram retiradas as distâncias dos pontos para usar nos cálculos teóricos 
de modo a ser possível comparar os valores teóriocos para path loss, potência recebida para três casos possíveis, antena a emitir a 250mW, 2W e a 20W, por
fim foram calculados os respetivos valores de SNR para dois valores típicos de Noise Floor, 95dBm e 105dBm.
Como o ficheiro é bastante extenso, este irá encontrar-se em anexo com os respetivos resultados, nesta subsubsecção iremos mostrar como foram obtidos os resultados
e apenas os que serviram de comparação para a próxima subsubsecção. A ordem dos pontos na tabela está de acordo com os nomes da figura \ref{fig:Pontos},
sendo que a ordem é Avenida do 7 -> entre Ed. X e Ed.7 -> Piso 1 Ed.X. 



\subsubsection{Teste e medições práticas}




\subsubsection{Comparação e analíse dos resultados teóricos com os resultados práticos}







\subsubsection{SNR necessário}

Como referido no capítulo anteriror, atrvés do software em Matlab, foi possível concluir quais os valores de SNR que devem ser garantidos para as modulações QPSK, 64-QAM e 256-QAM, 
de forma a ober um desempenho de BER  de $10^{-4}$, para canais AWGN e Rayleigh com e sem diversidade na receção. 

Os gráficos seguintes irão representar os valores de SNR necessários para as especificações dadas, a linha azul representa o valor teórico para o canal AWGN, a linha azul com asteriscos
é o valor teórico para o canal de Rayleigh e a linha verde é o resultado da simulação realizada.


\begin{table}[h]
    \centering
    \caption{Relação SNR necessária para diferentes esquemas de modulação e diversidade em canais AWGN e Rayleigh garantido um BER de $10^{-4}$.}
    \label{tab:performance_modulacao} 
    
    \vspace{0.2cm} 
\begin{tabular}{cccc}
\toprule
\textbf{Canal} & \textbf{Modulação} & \textbf{SNR} & \textbf{Diversidade} \\
\midrule
AWGN & QPSK   &  dB  & 1 \\
AWGN & QPSK   &  dB  & 4 \\
RAYL & QPSK   &  dB & 1 \\
RAYL & QPSK   &  dB  & 4 \\
\midrule
AWGN & 64-QAM &  dB & 1 \\
AWGN & 64-QAM &  dB & 4 \\
RAYL & 64-QAM &  dB & 1 \\
RAYL & 64-QAM &  dB & 4 \\
\midrule
% Valores estimados para 256-QAM baseados em tendências teóricas
AWGN & 256-QAM &  dB & 1 \\
AWGN & 256-QAM &  dB & 4 \\
RAYL & 256-QAM &  dB & 1 \\
RAYL & 256-QAM &  dB & 4 \\
\bottomrule
\end{tabular}
\end{table}


\textcolor{red}{Justificar a tabela com os valores preenchdios na tabela}

\subsubsection{Diversidade e diferença de canais}

\subsubsection{Densidades de energia por m2, a potência recebida e o noise floor requerido
no receptor}


\subsection{Fase 2}

Nesta etapa, com o objetivo de avaliar o impacto da não-linearidade do 
amplificador na qualidade de transmissão, procedeu-se à análise das curvas de BER em função do SNR 
para a modulação QPSK, 64-QAM e 256-QAM em ambos os canais com e sem diversidade. O desempenho foi avaliado considerando diferentes fatores de suavidade do amplificador, 
nomeadamente $p=1$, $p=2$, $p=10$ e $p=50$, estes valores foram escolhidos para ser possível uma maior comparação entre o pior caso $p=1$ onde iremos ver muita distroção e $p=50$
numa zona quase linear. Para estes ensaios, a tensão de saturação ($S_{sat}$) foi fixada em 1 de modo 
a forçar o amplificador a operar perto da saturação, tornando mais notórias as diferenças entres o valores de $p$.

