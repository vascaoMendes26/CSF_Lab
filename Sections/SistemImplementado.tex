\section{Sistema Implementado}

\subsection{1ª Fase}

\subsubsection{Análise teórica}

Na fase inicial deste trabalho foram escolhidos 3 pontos distintos, em 3 zonas dentro do campus
distintas(Avenida principal, entre Ed. X e Ed. 7 e dentro do piso 1 Ed. X), onde é possível ver na figura
\ref{fig:Pontos}.

\begin{figure}[htp]
    \centering
    \includegraphics[width=10cm]{Images/PontosAnalise.png}
    \caption{Pontos marcados no Google Earth}
    \label{fig:Pontos}
\end{figure}

Após terem sido marcados os pontos no mapa, com a ferramenta de medir distância 
do Google Earth, medímos a distância de cada ponto à antena colocando essas métricas num 
ficheiro excel para ser usado nos cálculos das equações \ref{eq:LFSPL} e \ref{eq:pr}.

Foram usados dois modelos para calcular o Path Loss: 
Para os 800 MHz é o COST 231, simples mas não tem em conta atenuações para os 3.4 GHz foi usado o alpha-beta-gamma(ABG),
este que é um pouco mais complexo, mas tem como grande vantagem a aplicação de coeficientes distintos($\alpha,\beta, \gamma$) dependendo se o recetor se encontra
em condição de Line of Sight(LOS) ou obstruído (NLOS), obtendo assim uma maior precisão perante obstáculos físicos.


Numa próxima etapa desta fase foi desenvolvido um simulador em Matlab para um sistema baseada em OFDM (Modulação por Divisão de Frequência Ortogonal),
utilizando 512 subportadores e outro sistema baseado em SC-FDE (Portadora Única com Equalização no Domínio da Frequência). 
Ambos os sistemas permitem analisar a propagação dos sianis codificados com esquemas de modulação QPSK, 64-QAM e 256-QAM e para canais AWGN (Ruído Gaussiano Branco Aditivo) e canais com fadding de Rayleigh, este que representa o pior caso para comunicação.
Além disto foi simulada a diversidade espacial na recepção, utilizando 4 antenas.
Nas Figura \ref{fig:BlocosOFDM}  e \ref{fig:BlocosSCFDE} é apresentado os diagramas de blocos do simulador tanto para OFDM quanto para SC-FDE onde é possível ver as diferenças entre ambos.

\begin{figure}[htp]
    \centering
    \includegraphics[width=10cm]{Images/BlocosOFDM.png}
    \caption{Diagrama de Blocos para o sistema implementado para um sistema OFDM}
    \label{fig:BlocosOFDM}
\end{figure}


\begin{figure}[htp]
    \centering
    \includegraphics[width=10cm]{Images/BlocosSCFDE.png}
    \caption{Diagrama de Blocos para o sistema implementado para um sistema SC-FDE}
    \label{fig:BlocosSCFDE}
\end{figure}


\subsubsection{Arquitetura do Sistema}

O sistema de transmissão implementado tem um estrutura de blocos dividida em Emissor, Canal e Recetor. Estes que são comuns em ambos os modos (OFDM e SC-FDE). O emissor gera os dados e a
modulação (mapeamento de bits para símbolos via QPSK, 64-QAM ou 256-QAM). Após isso é colocado um Prefixo Cíclico (CP), retirando a interferência Inter-Simbólica (ISI), devido ao canal ser dispersivo.

\subsubsection{Canal de Comunicação}

\begin{itemize}
    \item Ruído: AWGN (Additive White Gaussian Noise), este que demonstra o melhor cenário possível, contendo apenas ruído térmico e adensidade espectral de potência é constante;
    \item Desvanecimento: Modelo de Rayleigh, modela um ambiente de propagação mais realista, onde existe ausência de linha de vista (LOS) entre o emissor e recetor. 
\end{itemize}


\subsubsection{Diferenciação: OFDM vs SC-FDE}

Como foi abordado na componente teórica da unidade curricular a principal destrinçaõ entre estas duas técnicas está no processamento do sinal 
nos domínios do tempo e da frequência, isso é possível ser visto nas Figuras \ref{fig:BlocosOFDM} e \ref{fig:BlocosSCFDE}.

\subsubsection{Análise de Desempenho}

Para avaliarmos a qualidade da comunicação iremos nos basear na análise de erros, comparando os bits transmitidos com os recebidos 
com isso podemos calcular o bit error rate (BER), tento uma meta de $10_{-4}$, analisando qual o valor de SNR necessário para obter este desempenho.

\subsection{2ª Fase}

Na segunda fase do trabalho, foi implementado um bloco de amplificação  no emissor do sistema
com a finalidade de analisar o impacto de diversos níveis de amplificação e quais os impactos que farão 
no SNR para as modulações e canais utilizados. Voltando para as Figuras \ref{fig:BlocosOFDM} e \ref{fig:BlocosSCFDE} 
o bloco de amplificação foi implementado logo após o bloco do prefixo ciclico. 
Este amplificador baseia-se no tipo SSPA(Solid State Power Amplifiear), a característica de conversão AM/AM é descrita na Equação \ref{eq:Rapp}.

Os valores de amplificação utilizados foram p=1, p=2, p=10 e p=50, estes valores foram escolhidos
com o objetivo de analisar a não linearidade do amplificador sendo que para o valor de p=100  este é muito próximo da zona linear.


