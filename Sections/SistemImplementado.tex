\section{Sistema Implementado}

\subsection{1ª Fase}

\subsubsection{Análise teórica}

Na fase inicial deste trabalho foram escolhidos 3 pontos distintos, em 3 zonas dentro do campus
distintas(Avenida principal, entre Ed. X e Ed. 7 e dentro do piso 1 Ed. X), onde é possível ver na figura
\ref{fig:Pontos}.

\begin{figure}[htp]
    \centering
    \includegraphics[width=10cm]{Images/PontosAnalise.png}
    \caption{Pontos marcados no Google Earth}
    \label{fig:Pontos}
\end{figure}

Após terem sido marcados os pontos no mapa, com a ferramenta de medir distância 
do Google Earth, medímos a distância de cada ponto à antena colocando essas métricas num 
ficheiro excel para ser usado nos cálculos das equações \ref{eq:LFSPL} e \ref{eq:pr}.

Foram usados dois modelos para calcular o Path Loss: 
Para os 800 MHz é o COST 231, simples mas não tem em conta atenuações para os 3.4 GHz foi usado o alpha-beta-gamma(ABG),
este que é um pouco mais complexo, mas tem como grande vantagem a aplicação de coeficientes distintos($\alpha,\beta, \gamma$) dependendo se o recetor se encontra
em condição de Line of Sight(LOS) ou obstruído (NLOS), obtendo assim uma maior precisão perante obstáculos físicos.





