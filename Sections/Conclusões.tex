\section{Conclusões}

Este trabalho permitiu-nos entender a relação entre os modelos teóricos (COST 231 e ABG) e as medições de campo, apesar das aproximações 
em alguns pontos ser bastante precisa, este trabalho mostrou-nos a complexidade introduzida pelo multipath entre outros fatores que alteram os resultados. Com esta análise
foi possível sedimentar o nosso conhecimento prático, algo que realçamos como um ponto muito positivo deste trabalho. 

Numa segunda parte vimos a diferença entre as potências de sinais entre as diferentes
modulações, reforçando uma vez mais o que aprendemos nas aulas teóricas. Além das modulações vimos como o canal se comporta com mudanças na diversidade espacial consegue combater
o desvanecimento para qualquer tipo de canal, tanto ideal e o outro com fadding onde a diversidade se mostrou uma boa soluçaõ e mtrouxe uma melhoria nos resultados obtidos.

Por fim a fase 2, destacou a vantagem do SC-FDE sobre o OFDM, onde colocámos duras restrições nos amplificadores, simulando amplificadores baratos e de baixa qualidade,
analisando com o canal se comporta, devido ás distroções não-lineares, foi muito complicado de receber tornou-se muito complicado de obter um BER de $10^{-4}$ para modulações mais altas.

Após esta análise do trabalho, vemos que o objetivo foi concluído com sucesso, onde obtívemos resultados positivos para
analise de cobertura e qualidade do sinal com comunicações sem fios, comparando com resultados teóricos e práticos, foi 
visto as diferenças e realizadas simulações para sistemas com e sem diversidade na recepção na presença de modulações multi e mono-portadora com e sem uso de amplificadores. 